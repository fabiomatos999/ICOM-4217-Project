\section{Theoretical Background}
The major considerations for laying out the requirements for this design were the state guidelines for insulin storage and the expiration dates for various manufacturers of insulin. According to the Centers for Disease Control (CDC)\cite{ManagingInsulinEmergency2022}, the Food and Drug Administration (FDA) \cite{researchInformationRegardingInsulin2018} and the National Institute of Health (NIH) \cite{bahendekaEADSGGuidelinesInsulin2019} along with our client, for long term storage vials of insulin ought to be stored between 2\si{\celsius} and 8\si{\celsius} and can have a shelf life of up to 2 years. With these criteria in mind, our team set of to create a design that was:
\begin{itemize}
  \item Inexpensive to manufacture
  \item Uses off the shelf parts
  \item Can operate in cold and hot environments
  \item Can be reused
  \item Easy to integrate in existing production lines
  \item Has a long battery life
  \item Can operate with/without the availability of mains power
  \item Ensured compliance with federal guidelines for insulin storage
\end{itemize}
In terms of logistics, the project had a timetable of less than 4 months from pre-proposal to final prototype. In addition, since this project was being developed in the University of Puerto Rico, Mayagüez Campus, all of the parts needed to construct the prototype needed to be shipped from overseas which constricts the timetable even further.\\
