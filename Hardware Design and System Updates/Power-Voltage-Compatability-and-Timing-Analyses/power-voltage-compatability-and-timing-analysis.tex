\section{Power, voltage compatibility,\\ and Timing Analyses}
\subsection{Main Unit PSU}
According to the TCT50-01E07AB data-sheet \cite{TCT5001E07AB}, given an input voltage of 120$\si{\V}_{RMS}$ AC voltage, it will output a 10$\si{\V}_{RMS}$ AC voltage, thus $N\$1$ and $N\$2$ will have a voltage of -10\si{\V} to 10\si{\V}, depending on which part of the sine wave is currently being supplied. The voltage at the output of the bridge rectifier B1 at $N\$8$ was calculated as follows:
\begin{equation}
  B1_{Out} = N\$1 - V_{BD}
  \label{eq:main-b1out}
\end{equation}
In this design B1 has a drop of 0.55 per element and since there are 2 elements per cycle this means $V_{BD}$ is equal to 1.1\si{V} \cite{KMB23S}. This means that equation \ref{eq:main-b1out} is 22.9\si{\V} for this design. Then the voltage at $N\$8$ is rectified to 22.8\si{V} by D2 (a 22.8\si{\V} zener diode) and is denoised by C1 (a 50\si{\V} 470\si{\micro\farad} capacitor) which was calculated as follows:
\begin{equation}
  V_{PP} = \frac{I}{2fC}
  \label{eq:main-filter-cap}
\end{equation}
$V_{PP}$ is the peak-to-peak ripple voltage, $I$ is the current in the circuit, $f$ is the source frequency of the AC power and $C$ is the capacitance of the selected capacitor. Given that equation \ref{eq:main-b1out} is equal to 22.8 in this design, a ripple voltage of less than 1\% was chosen. To simplify the equations, the in operation current for the Bluetooth module was chosen to represent $I$ which according to the HM-10 data sheet is 8.5\si{\mA} and the rest of the current consuming devices were assumed to be negligible. Given that $f$ is the mains frequency in Puerto Rico and the United States of America, $f$ is 60\si{\Hz}. Solving for C given $\%V_{PP}$ gives the following equation:
\begin{equation}
  C = \frac{I}{2f\%V_{PP}}100
  \label{eq:cap-given-percent-vpp}
\end{equation}
For this design, this equation solves to 310\si{\micro\farad}, which assuming $I$ is accurate a 330\si{\micro\farad} capacitor would suffice but since it was assumed that other current drawing sources besides the Bluetooth module were negligible, this means that $I$ is an underestimate of the true current through the system. Thus a 470\si{\micro\farad} was chosen to account for the underestimation in current.\\\\
$N\$8$ is then fed to $+VIN$ which is compatible with the input voltage since $+VIN_{Min} < N\$8 < +VIN_{Max}$ which are 11\si{\V} and 32\si{\V} respectively \cite{R78W900}.
$N\$11$ was calculated as follows:
\begin{equation}
  N\$11 = DC3_{+VOUT}
  \label{eq:main-N11}
\end{equation}
In this case since $+VOUT$ is the output of a 9\si{\V} DC-DC converter, thus $N\$11$ is at 9\si{V} DC. Since $N\$11$ is at 9\si{V}, $DC2_{+VIN}$ is compatible since $+VIN_{Min} < N\$11 < +VIN_{Max}$ which are 4.75\si{\V} and 36\si{\V} respectively. $N\$9$ is the output of DC2 (a 3.3\si{\V} regulator) and thus will be at 3.3\si{\V} which U2 is compatible since $N\$9$ is less than $VIN_{Max}$ for the BA033CC0FP-E2 of 2.5\si{\V} \cite{BA033CC0FPE2}. Then $N\$3$ will be at 3.3\si{\V} denoised and will serve as the GPIO input mains power sense pin for the MCU.
Going back down to $N\$5$, when mains power is available the voltage at this node is calculated as follows:
\begin{equation}
  N\$5_{Mains} = N\$11 - V_{D}
  \label{eq:main-N5-mains}
\end{equation}
To understand why equation \ref{eq:main-N5-mains} is calculated this way, we need first see how BT1, BT2 and D3 interact with each other. BT1 and BT2 are both 3.7\si{\V} 18650 cells \cite{ICR186502200F} in series, thus the voltage at the input of D3 will 7.4\si{\V}. The diode drop for D3 is 1.0\si{\V} \cite{1N4004}, this means that the potential difference at $N\$6$ by the batteries is 6.4\si{\V}. The potential difference by $N\$11$ is 8\si{\V}, thus the voltage difference between the output of DC3 minus $V_{D}$ of D1 and the voltage of the batteries minus $V_{D}$ is 1.6\si{\V}. This means that the D3 will be placed in reverse bias and therefor the batteries will not conduct. When mains power is available, $DC1_{+VINMin} < N\$5 < DC1_{+VINMax}$, this making it compatible with DC1 \cite{K7803500R3}. Since $N\$4$ is the same as $N\$9$, U1 is compatible. The output of U1 ($N\$12$) will serve as the VCC for the MCU.
When mains power is not available, initially $N\$5$ but instantaneously D3 is placed in forward bias and the batteries can conduct, and as calculated above $N\$5$ will be at 6.4\si{\V} which is sufficient to power DC1 and U1.\\\\
At $N\$3$ and $N\$12$ a 0.1\si{\micro\farad} 50\si{\V} capacitor were placed in parallel with the load as a bypass capacitor in order to ensure a clean DC source at both VCC and power sense pin of the MCU. The value of 0.1\si{\micro\farad} was chosen since it is a common value for bypass capacitors for microcontrollers.
\subsection{Sub Unit PSU}
$N\$6$ will be at $2V_{BATT}$, this is due to BT1 and BT2 being in series and each cell is 3.7\si{\V} \cite{ICR186502200F} and thus $N\$6$ will be at 7.4\si{\V}. Since $N\$6$ is at 7.4\si{\V}, it is in the compatible voltage range for DC1 \cite{BA033CC0FPE2}. At $N\$4$ the voltage will be regulated by DC1 and thus will be at 3.3\si{\V}, not exceeding the maximum voltage rating for U1 \cite{BA033CC0FPE2}. $N\$1$ will serve as the regulated output for the MCU.
\subsection{Sub Unit Voltage \& Baud Rate}
In order to validate the compatibility of MCU MSP430FR2000IPW16 and Bluetooth module HM-10S with respect to their voltage levels and communication protocols, it is necessary to verify their respective related specifications.\\\\
The MSP430FR2000IPW16 component showcases flexibility in terms of voltage adaptability, functioning within a supply range stretching from 1.8V to 3.6V as seen in page 1 from it’s datasheet [CITE HERE]; on the other hand, the HM-10S Bluetooth module has a good operational capacity that ranges from 2.5Volts all the way up to 3.3Volts as shown in page 1 of the HM-10S datasheet [CITE HERE]. Knowing this, it will be needed to power this module via a 3.3V power source so as to satisfy both the voltage requirements for operation alongside those required by MSP430FR2000IPW16 device compatibility guidelines.\\\\
The MSP430FR2000IPW16 is designed to be compatible with various baud rates for serial communication via its Universal Asynchronous Receiver / Transmitter (UART) interface. With a capability of generating 9600 up to 115200 bits per second (bps), it can facilitate efficient data transmission. The HM-10S Bluetooth module also uses UART, and the supported range of baud rates from the datasheet of the HM-10S Bluetooth module is within the capabilities of the MSP430FR2000IPW16 micro-controller, hence compatibility exists between them at this level.\\\\
Another aspect that needs to give thoughts to is the voltage analysis between the MSP430FR2000IPW16 MCU and the SSL-LX3052ID Red LED. Here the voltage requirements for both devices are considered by taking into account their respective voltage needs.\\\\
As previously said, the MSP430FR2000IPW16 operates at a supply voltage range of 1.8V to 3.6V. The voltage requirement of the SSL-LX3052ID Red Led has a forward Voltage of 2.0 and a maximum continuous forward current of 30mA. Therefore, it has to be ensured that the LED is powered with a voltage that meets its forward voltage requirement and a current that does not exceed its maximum forward current rating. \\\\
To meet these requirements, it has to be considered how the LED is connected to the MCU. To control forward Voltage intensity and the current flowing through the LED, a resistor will be needed between both. To know the R needed for this resistor, the following formula was used:\\\\
\begin{equation}
R = \frac{V_{cc} - V_{f}}{I_{f}}
\end{equation}
Where Vcc is the Imput Voltage (3.3 V), Vf the Forward Voltage of the LED(2.1 V) and If the Forward Current of the LED (10 mA). Therefore, a 120\si{\ohm} resistor will be needed in the circuit to limit the current flowing through the LED to 10mA.  \\\\
