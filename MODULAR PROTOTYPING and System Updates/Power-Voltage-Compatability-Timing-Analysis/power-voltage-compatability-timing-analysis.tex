\section{Power, Voltage Compatibility and Timing Analysis}
\subsection{Main Unit PSU}
According to the RAC03-12SK data sheet\cite{RAC0312SK}, the AC/DC converter can handle from 85\si{\V}AC to 264\si{\V}AC and produce 12\si{\V}DC at its output. Since this design will be tested in a 120\si{\V}AC electrical system, it is compatible.\\ At the output of the AC/DC converter, a 1N4004-G diode\cite{1N4004G} and a R78-E3.3-0.5 3.3\si{\V} DC/DC converter\cite{R78E330} (DC2) are placed in parallel. For DC2, according to its data sheet, is compatible within a voltage range of 6\si{\V} to 28\si{\V}. Since DC2 is at the output of the 12\si{\V} AC/DC converter, it is compatible.\\ At the node at the right of D1 (N\$5), when mains power is available is calculated as follows:
\begin{equation}
  V_{N\$5} = 12 -  V_{D}
  \label{eq:mains-batt-node-eqation-mains-available}
\end{equation}
where $V_{D}$ is the diode drop of the 1N4004-G which according to its data sheet is 1.1\si{\V}, resulting in $N\$5$ equal to 10.9\si{\V}. When there is mains power available, the node will be at 10.9\si{\V}, the batteries will be trying to conduct but since the voltage of the two 3.6\si{\V} cells in series will be 7.2\si{\V}, minus the diode drop of CR1\cite{1N5819} will be $7.2 - 0.55 = 6.65\si{\V}$. Therefore, the potential difference between mains power and battery power is $-4.25\si{\V}$, therefore CR1 will be placed in reverse bias and thus the batteries will not conduct, but since the negative potential difference is low, it is compatible, according to the 1N5819 data sheet\cite{1N5819}.\\ When mains power is available DC1, another K7803-500R3, will receive a voltage of 10.9\si{\V} and as mentioned above is compatible and will produce a 3.3\si{\V} output. At the outputs of both DC1,DC2 there are identical 3.3\si{\V} LDOs (BA33BC0T)\cite{BA33BC0T}, which according to their data sheet can accept up to 16\si{\V} which is compatible since U1 and U2 are being supplied by 3.3\si{\V} DC/DC converters and thus are compatible. At the outputs of U1 and U2 are C2 and C3 respectively, that are 0.1\si{\micro\farad} filtering capacitors whose voltage rating is 50\si{\V} which are compatible.\\ Going back to $N\$5$ DC4 (5\si{\V} DC/DC converter) is in parallel with DC1, thus when there is mains power available, will be supplied 10.9\si{\V}. According to the TSR 1-2450 (DC4) data sheet\cite{TSR12450} it can accept a DC voltage from 6.5\si{\V} to 36\si{\V} thus making it compatible and will output 5\si{\V} DC. At the output of DC4, a 5\si{\V} LDO is fed (L7805CV
), whose data sheet\cite{L7805CV} says that it can handle up to 36\si{\V} thus making it compatible since the output of DC4 is 5\si{\V}. Similarly to C3 and C4, a 0.1\si{\micro\farad} capacitor is placed in parallel with the output and is compatible since it has the same voltage rating of 50\si{\V}.\\ When mains power is not available DC2 will shutdown and D1 will be placed in reverse bias since CR1 will be conducting due to $N\$5$ will be at 0\si{\V} when mains power is not available and then will be placed at 6.65\si{\V}. Since DC1, DC2 and DC4 all have a minimum voltage less than 6.65\si{\V} they are all compatible and the LDOs and capacitors will operate as normal.
\subsection{Sub Unit PSU}
The sub unit PSU operates in a very similar fashion to the main unit PSU when mains power is not available. BT1 and BT2 supply a 3.3\si{\V} (K7803-500R3) \cite{K7803500R3} and a 5\si{\V} (TSR 1-2450) \cite{TSR12450} DC/DC converter in parallel. As analyzed in the main unit PSU, both DC/DC converters require less than 6.5\si{\V} to conduct properly and since the battery is directly supplying the DC/DC converters, they will receive a nominal voltage of 7.2\si{\V} since BT1 and B2 are 3.6\si{\V} cells in series.\\ U1 (BA33BC0T) \cite{BA33BC0T} U2 (BA33BC0T) will also be compatible as analyzed above since their maximum input are above 3.3\si{\V} and 5\si{\V} respectively. As for C1 and C3, they have a voltage rating of 50\si{\V}, therefore are compatible being in parallel with the outputs of U1 and U2.
\subsection{Main Unit}
The Main Unit consists of the MSP430FR2110IPW16 MCU and the HM-10 Bluetooth module. For the MCU, it has an input voltage range from 1.8\si{\V} to 3.6\si{\V} according to it's data sheet \cite{MSP430FR2110IPW16R}. Meanwhile, the HM-10 Bluetooth module has an input voltage from 3.6\si{\V} to 6\si{\V} according to it's product listing \cite{AmazonComHiLetgo}. Due to this the MSP430FR2110 has to be supplied by the 3.3\si{V} output of the PSU and the HM-10 has to be supplied by the 5\si{\V} output of the power supply.\\
In order to connect both devices to each other, for communication, the best protocol to use will be UART. The MSP430FR2110 has 4 pins for UART communication using the eUSCI modules, which can detect automatically the baud-rate to be used. According to the datasheet for the HM-10 module\cite{AmazonComHiLetgo}, it can select the baud-rate to be used, which ranges from a minimum 1200 to a maximum of 230400. In the datasheet of the MSP430FR2110, the eUSCI clock frequency, UART mode, can reach a maximum of 5MHz, which means it can reach the maximum baud-rate of the Bluetooth module and can also select lower baud-rates, according to the formula:
\begin{equation}
	Clk Frequency = Baudrate \cdot 16
	\label{eq:UART Frequency}
\end{equation}
This means if we select the maximum baud-rate of the HM-10 module, which is 230400, using the formula, we would need a frequency of 3.68MHz, which is still below the maximum frequency the eUSCI module on the MSP430FR2110. This will allow us to be able to use the HM-10 communicate via UART with the MSP430FR2110.\\
For the power available pin, according to the MSP430FR2110 data sheet, the GPIO pins can handle up to VDD =0.3\si{\V}, since in this design VDD is 3.3\si{\V} and the power available pin is at 3.3\si{\V}, it is compatible with the power sense input.\\
In order to calculate the resistor values for the green and red LEDs we used the following formula:
\begin{equation}
R = \frac{V_{cc} - V_{f}}{I_{f}}
\end{equation}
Where VCC is 3.3\si{\V} $V_{f}$ is the forward voltage for each LED and $I_{f}$ is the forward current for each LED. For the green LED \cite{GreenDiffused5mmStandard} it has a forward voltage of 2.25\si{\V} and since the MCU has a max I/O current of 2\si{\mA}, this would mean that it requires a 525\si{\ohm} resistor. Similarly, for the RED LED, it has a 2\si{\V} forward voltage and a max I/O current of 2\si{\mA}, this gives us a resistor value of  650\si{\ohm}.\\\\
Finally, for the temperature sensor (U5) \cite{TMP36GT9Z} and operational amplifier (U4) \cite{MCP6022I}, they both can be supplied by a 3.3\si{\V} supply and thus are compatible. Since the temperature sensor range for this application will be from 0\si{\celsius} to 40\si{\celsius}, at a resolution of 0.1\si{\celsius}, since there is an offset of 0.5\si{\V} by the temperature sensor gives us a range of 900 unique values so that the temperature sensor can reach its full negative temperature range. At 40\si{\celsius} the temperature sensor will be at 900\si{\milli\volt}, setting the ADC to use a 1.5\si{\V} reference. gives us a $A_{v}$ of 1.666. Setting up the op amp in a non inverting configuration, using 1\si{\kilo\ohm}, 620\si{\ohm}, 39\si{\ohm} and 5.1\si{\ohm} resistors will yield a $A_{v}$ of 1.6642.
\subsection{Sub Unit}
In order to validate the compatibility of the MCU (MSP430FR2000IPW16) and Bluetooth module (HM-10S) with respect to their voltage levels and communication protocols, it is necessary to verify their respective related specifications.\\
The MSP430FR2000IPW16 showcases flexibility in terms of voltage adaptability, functioning within a supply range ranging from 1.8V to 3.6V as seen in page 1 of its data sheet \cite{MSP430FR2000IPW16}. On the other hand, the HM-10 Bluetooth module has an input voltage from 3.6\si{\V} to 6\si{\V} according to it's product listing \cite{AmazonComHiLetgo}. Due to this the MSP430FR2110 has to be supplied by the 3.3\si{V} output of the PSU and the HM-10 has to be supplied by the 5\si{\V} output of the power supply.\\
In order to connect both devices to each other, for communication, the best protocol to use will be UART. The MSP430FR2000 has 4 pins for UART communication using the eUSCI modules, which can detect automatically the baud-rate to be used. According to the datasheet for the HM-10 module\cite{AmazonComHiLetgo}, it can select the baud-rate to be used, which ranges from a minimum 1200 to a maximum of 230400. In the datasheet of the MSP430FR2000, the eUSCI clock frequency, UART mode, can reach a maximum of 5MHz, which means it can reach the maximum baud-rate of the Bluetooth module and can also select lower baud-rates, according to the formula:
\begin{equation}
	Clk Frequency = Baudrate \cdot 16
\end{equation}
This means if we select the maximum baud-rate of the HM-10 module, which is 230400, using the formula, we would need a frequency of 3.68MHz, which is still below the maximum frequency the eUSCI module on the MSP430FR2000. This will allow us to be able to use the HM-10 communicate via UART with the MSP430FR2000.\\
In order to calculate the resistor values for the green and red LEDs we used the following formula:
\begin{equation}
R = \frac{V_{cc} - V_{f}}{I_{f}}
\end{equation}
Where VCC is 3.3\si{\V} $V_{f}$ is the forward voltage for each LED and $I_{f}$ is the forward current for each LED. For the RED LED \cite{SSLLX3052ID}, it has a 2\si{\V} forward voltage and the MSP430FR200 has a max I/O current of 2\si{\mA}, this gives us a resistor value of  650\si{\ohm} in order to drive this LED successfully and within MCU maximum parameters.\\
Finally, for the temperature sensor (U5) \cite{TMP36GT9Z} and operational amplifier (U4) \cite{MCP6022I}, they both can be supplied by a 3.3\si{\V} supply and thus are compatible. Since the temperature sensor range for this application will be from 0\si{\celsius} to 40\si{\celsius}, at a resolution of 0.1\si{\celsius}, since there is an offset of 0.5\si{\V} by the temperature sensor gives us a range of 900 unique values so that the temperature sensor can reach its full negative temperature range. At 40\si{\celsius} the temperature sensor will be at 900\si{\milli\volt}, setting the ADC to use a 1.5\si{\V} reference. gives us a $A_{v}$ of 1.666. Setting up the op amp in a non inverting configuration, using 1\si{\kilo\ohm}, 620\si{\ohm}, 39\si{\ohm} and 5.1\si{\ohm} resistors will yield a $A_{v}$ of 1.6642.
