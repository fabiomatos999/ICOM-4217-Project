\documentclass[12pt]{article}
\usepackage{setspace}
\usepackage[T1]{fontenc}
\usepackage{helvet}
\linespread{2.0}

\begin{document}

\section{Cardiac Monitoring Warning System}

The design I have in mind uses a micro-controller to verify that there is nothing unusual with the cardiac system of a patient. This design would be useful in situations were patients need a constant monitoring of their heartbeats. The end goal of this design is to create a prototype using a micro-controller, detect if there's an anomality with the heartbeats of the patient using a heartbeat sensor. The patient will be able to carry this micro-controller at any time to keep monitoring their heartbeats at all times. It will have a display for the patient to see the results of the current heartbeats test and a log in a removable storage for previous tests that the design will be able to run in time intervals. If the micro-controller detects an anomality of the heartbeats it will notify the user by an alarm / warning letting know that something is not right. For example, if the heartbeats are too fast a beep will sound rapidly 3 times. If the opposite occurs, then the alarm will be a slow beep 3 times. This problem will need an economic design with low power consumption to facilitate the patient its use. Some features of this design are, a real time clock for keeping the logs and to measure the heartbeats, a small display for the user to be able to know their heartbeats levels, a compact battery for the energy consumption, a micro USB port to retrieve the logs of the patient, a heartbeats sensor and a buzz alarm for the warnings. The end goal for this proposal will be a design that uses a micro-controller, a heartbeat sensor, a small display, and a removable micro SD storage to log, monitor and inform the patient any issues related to their heartbeats.


\newpage
\section{Who is your Client? Guillermo}
Client: Pedro Bernardi Ortiz\\
Position: Heart operated patient\\
Contact Information: (787) 735-4584\\
\\
Client: José G. Colón Colón\\
Position: Pacient with high blood preasure\\
Contact Information: (787) 566-2820\\

\end{document}