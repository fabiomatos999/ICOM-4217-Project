\pagestyle{plain}
\chapter{}
The design I have in mind uses a micro-controller to verify if stored insulin meets the normal and emergency guidelines for insulin storage. This design would be useful in regions where the power infrastructure is unreliable and post disaster conditions are frequent. The end goal of this design is to create a prototype using a micro-controller, detect the current temperature using a temperature probe and log to a removable storage the recorded temperatures the at specific time intervals. If the micro-controller detects there is AC power being supplied from the electrical grid it verifies if the insulin meets the normal guidelines for insulin storage, i.e that the insulin is less than 2 months in storage and has been kept consistently between 2 and 8 Celsius. When AC power has been lost, the micro-controller switches to the emergency guidelines for insulin storage, i.e 15 to 30 Celsius for less than 4 weeks. The time sensitive nature of this problem necessitates a low power solution, that can make decisions on the fly while maintaining an accurate real-time clock. Some of the features include An uninterruptible power supply in order switch from AC to battery power when AC power is lost, micro SD port to log the temperatures registered by the sensor and a display to show the remaining time of the insulin still meeting the guidelines and a buzzer to alert the user if insulin has been possibly frozen and/or exposed to high temperatures. The end goal for this proposal will be a design that uses a micro-controller, a temperature sensor, a segmented display and removable micro SD storage to log, monitor and inform the user if the insulin has been possibly outside the safety guidelines.
\newpage
\section{Insulin Temperature Warning System Client}
Kali Villegas (Temperature sensitive medication patient) 787-669-1130.
