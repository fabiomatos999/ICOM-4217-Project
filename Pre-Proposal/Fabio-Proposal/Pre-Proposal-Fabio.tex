\documentclass[12pt]{article}
\usepackage{unicode}
\usepackage{setspace}
\usepackage[T1]{fontenc}
\usepackage{helvet}
\linespread{2.0}
\title{ICOM 4217 Pre-Proposal}
\author{Fabio J. Matos Nieves\\
  Enrique Chompré\\
  Guillermo Colón\\
Rubén Marrero}
\date{24 January 2023}
\begin{document}
\maketitle
\newpage
\thispagestyle{empty}
\pagestyle{plain}
\section{Team Members}
\pagenumbering{arabic}
\begin{itemize}
  \item Fabio Matos
  \item Enrique Chompré
  \item Guillermo Colón
  \item Rúben Marrero
\end{itemize} 
\newpage
\section{Potential Project Description Fabio}
The design I have in mind uses a micro-controller to detect, log and alert a user if a refrigeration unit (i.e a refrigerator or freezer) has experienced a power outage, for how long and if a network is available, alert a user if the temperature inside the refrigeration unit has surpassed a set threshold. The setting in which this proposed design would be useful is in areas where power outages are very common in areas with outdated, non-maintained and fragile electrical infrastructure like in rural areas of Puerto Rico and many other poorer countries around Central and South America. The need for this design would be to ensure food and drug safety in the case of extended and sudden blackouts. The motivating factor is that in the case of Puerto Rico the unreliable electrical infrastructure has led much of the public in a constant of fear that medicine like insulin (which Puerto Rico has one of the largest population of diabetics the USA) are temperature sensitive and can spoil without proper and constant refrigeration, in fact, spoiled medicine can be extremely dangerous and can lead to infections and other serious health complications. The same can be said of refrigerated consumables than can spoil much quicker without refrigeration which can be especially dangerous in post natural disaster conditions like in hurricane Maria and Fiona.
\newpage
\section{Who is your client (position and contact information)? Fabio}
Carmen Perez Muños (Insulin Patient) 939-464-9542
\end{document}
