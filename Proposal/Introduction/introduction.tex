\chapter{Introduction}
In 2017 hurricane Maria devastated the northeastern Caribbean, particularly Dominica, Saint Croix, and Puerto Rico. In the wake of the storm, Puerto Rico suffered the longest blackout in US history and the second longest blackout in recorded human history [1]. The revised death count attributable to the hurricane was 2,975 people [2]. According to an investigation by the New York Times, deaths caused by  sepsis, pneumonia, emphysema, diabetes, and Alzheimer's and Parkinson's spiked in the two months following the hurricane.[3]. The whole island of Puerto Rico was declared a disaster zone by the Federal Emergency Management Agency (FEMA) and it wasn't until four to eight months after the storm that most of the population had electricity and running water after the storm. The Puerto Rican electrical grid also collapsed in the aftermath of hurricane Fiona in September of 2022. Puerto Rico also has a sizable portion of the adult population with diabetes and a large portion of those diagnosed with diabetes require insulin to survive [4]. The rationale for this project is to provide a tool for users of insulin to be sure that their medication is safe and effective in normal and post-disaster conditions. The problem is that there has not been a way to monitor the storage conditions of insulin in a consumer setting in normal/post-disaster conditions. This project's solution to this problem is to create a micro-controller based device that can:
\begin{itemize}
  \item Monitor the temperature of the insulin, making sure that is safe to administer to the patient.
  \item Check if an extended power-outage has occurred.
  \item Decide wether to use the normal or emergency storage guidelines for insulin
  \item Warn the user is the insulin has been frozen or overheated.
  \item Display current the current temperature of the insulin and remaining time for safe administration.
\end{itemize}
