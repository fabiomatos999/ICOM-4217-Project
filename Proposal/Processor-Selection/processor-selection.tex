\section{Processor Selection}
Based on the global system view, in terms of input output devices (I/O) the sub-unit just requires 1 Bluetooth module and an LED. A cursory glance on the internet reveals that a HC-05 Bluetooth module\cite{InterfacingHC05Bluetooth} uses two I/O and one I/O pin for controlling the LED. The main requires the same two pins for the Bluetooth module and one for detecting power outages. This means that both units need a measly three I/O pins and according to the HC-05 data sheet, the default baud rate is 38400 baud, which translate to 38.4 KHz which most MCU can easily obtain. There is no intense requirements for computation or memory, but peripherals are required, thus it is justified that an MCU be used for this design. However, the MCU has to be very low power since the sub-units only operate on batteries and have to last at least two years due to the fact that refrigerated vials of insulin can last upwards of a year in ideal conditions.
\begin{enumerate}
  \item MSP430F5515IPN: It has 63 I/O, a real time clock (RTC) and 12 bit ADC with a unit price of \$5.50. It also has a development kit and IDE for rapid prototyping.
  \item STM32F100RCT6B: It has 51 I/O, also has an RTC and a 12 bit ADC with a unit price of \$7.59. It also has an IDE, official libraries and development kit similar in price to the MSP430 but with a unit price almost 50\% more expensive than the MSP430.
  \item PIC24FJ256GA106-I/PT: It has 64 I/O, also has an RTC, a 16 bit ADC with a unit price of \$7.57. It also has a development kit and IDE but the development kit is almost three times more expensive than the MSP430 and per unit price it is almost 50\% more expensive.
\end{enumerate}
With this design being focued on low cost due to mass production, the MSP430F5515IPN was chosen due to it's lowest price per unit, IDE and development kit availability.
