\section{Specifications}
The major requirements, features and limitations for hardware and software for this design are as follows:
\begin{itemize}
  \item Sub-Unit:
        \begin{itemize}
          \item Hold the current time, expiration date and lot number in memory.
          \item Check the current temperature using the on-board temperature sensor.
          \item Check that the current temperature meets safe storage guidelines.
          \item Check that the current time is less than the expiration date
          \item Send an appropriate error code if either expiration date has been passed or the current temperature is not within the guidelines.
                \item Send lot number information to the main-unit.
        \end{itemize}
  \item Main-Unit:
        \begin{itemize}
          \item Check that the current temperature meets safety guidelines for safe storage.
          \item Receive error codes from the sub-units.
          \item Forward error codes to an application that an operator can take appropriate action on.
          \item Check if mains power is available.
          \item Log the current temperature power, status and time to a log file and periodically forward it to a server.
          \item Alert an operator via an app if the temperature does not meet the safety guidelines or that power is not available.
          \item Send logs of lots of sub-units that entered the main-units range.
        \end{itemize}
\end{itemize}
This design satisfies all four essential component for eligibility as follows:
\begin{enumerate}
  \item \underline{Communications}: This design communicates 2 ways:
        \begin{itemize}
               \item The sub-units sends their lot numbers to the main-unit once they are in range and the main-unit forwards that information to a server.
               \item The sub-units send error codes to the main-unit and then the main unit sends the error information to an application where an operator can take further action based on the information provided.
        \end{itemize}
  \item \underline{User Interface}: For the sub-units the only way that it interacts with a user is via an LED, warning the user if an error status has occurred. The main unit, interacts with a user via the sent logs being displayed by an app and warning notifications if a critical error has occured.
  \item \underline{Control Scheme}: The main way that this design controls something, is that it notifies an operator to take action if a sub-unit sends an error and then can take appropriate action the box on insulin.
  \item \underline{Microprocessor-based}: This design meets the needs for it to be microprocessor based in that we need the sub-units to be low cost, and exceedingly power efficient without needing access to a lot of RAM or storage but still needing to have I/O. The main unit also necessitates it based due to not having to compute much, needing to have I/O and relatively low requirements for memory and storage and needing it being inexpensive.
\end{enumerate}
